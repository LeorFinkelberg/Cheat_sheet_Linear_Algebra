\documentclass[%
	11pt,
	a4paper,
	utf8,
	%twocolumn
		]{article}	

\usepackage{style_packages/podvoyskiy_article_extended}


\begin{document}
\title{Сборник заметок по линейной алгебре и сопряженным вопросам}

\author{\itshape Подвойский А.О.}

\date{}
\maketitle

\thispagestyle{fancy}


%\shorttableofcontents{Краткое содержание}{1}


\tableofcontents

\section{Система $ m $ линейных алгебраических уравнений с $ n $ неизвестными}

Матричная запись неоднородной системы уравнений имеет вид
\vspace*{-3mm}
\begin{align*}
	A x = b,
\end{align*}

\vspace*{-3mm}
а однородной
\vspace*{-3mm}
\begin{align*}
	A x = o,
\end{align*}

\vspace*{-3mm}
где $ o $ в правой части обозначает нулевой столбец размеров $ m \times 1 $.

Эту матричную запись неоднородной системы уравнений можно представить в эквивалентной форме
\begin{align*}
	\begin{pmatrix}
		a_{11} \\
		\vdots \\
		a_{m1}
	\end{pmatrix} x_1 + 
    \begin{pmatrix}
    	a_{12} \\
    	\vdots \\
    	a_{m2}
    \end{pmatrix} x_2 + \ldots +
    \begin{pmatrix}
    	a_{1n} \\
    	\vdots \\
    	a_{mn}
    \end{pmatrix} x_n = 
    \begin{pmatrix}
    	b_1 \\
    	\vdots \\
    	b_m.
    \end{pmatrix}
\end{align*}

Тогда решение системы представляется столбцом
\begin{align*}
	x =
	\begin{pmatrix}
		\alpha_1 \\
		\vdots \\
		\alpha_n
	\end{pmatrix}
\end{align*}
и удовлетворяте равенству
\begin{align*}
	\begin{pmatrix}
		a_{11} \\
		\vdots \\
		a_{m1}
	\end{pmatrix} \alpha_1 +
    \begin{pmatrix}
    	a_{12} \\
    	\vdots \\
    	a_{m2}
    \end{pmatrix} \alpha_2 + \ldots + 
    \begin{pmatrix}
    	a_{1n} \\
    	\vdots \\
    	a_{mn}
    \end{pmatrix} \alpha_n =
    \begin{pmatrix}
    	b_1 \\
    	\vdots \\
    	b_m,
    \end{pmatrix}
\end{align*}
т.е. столбец свободных членов $ b $ является линейной комбинацией столбцов матрицы системы.




\section{Теорема (правило) Крамера}

Система называется \textbf{совместной}, если она имеет \emph{хотя бы одно решение}. Система называется \textbf{несовместной}, если она \emph{не имеет ни одного решения}.

Если определитель $ \Delta = \det A $ матрицы системы $ n $ линейный независимых уравнений с $ n $ неизвестными отличен от нуля ($ \det A \neq 0 $), то система имеет \emph{единственное} решение, которое находится по формулам
\begin{align*}
	x_i = \dfrac{ \Delta_i }{ \Delta }, \ i = 1,\ldots, n, \quad (\Delta = \det A \neq 0),
\end{align*}
где $ \Delta_i $ -- определитель матрицы, полученной из матрицы системы $ A = [a_{ij}]_{i,j=1}^n $ заменой $ i $-ого столбца столбцом свободных членов.

ЗАМЕЧАНИЕ: на практике при больших $ n $ правило Крамера не применяется!

Если $ \Delta = 0 $ (матрица коэффициентов системы вырождена) и хотя бы один определитель $ \Delta_i \neq 0 $, то система \emph{несовместна}, т.е. не имеет ни одного решения. Если же $ \Delta = \Delta_1 = \Delta_2 = \ldots, \Delta_n = 0 $, то возможны два случая: либо система несовместна (не имеет ни одного решения), либо система имеет бесконечно много решений \cite[\strbook{188}]{bortakovskiy:2005}.

\section{Условие совместности системы линейных уравнений. Теорема Кронекера-Капелли}

Рассмотрим систему $ m $ линейных уравнений с $ n $ неизвестными. Составим блочную матрицу, приписав к матрице $ A $ справа столбец свободных членов $ b $. Получим \emph{расширенную матрицу системы}
\begin{align*}
	\underset{m \times (n + 1)}{(  A \ | \ b )} =
	  \begin{pmatrix}
	  	  a_{11} & \dots & a_{1n} &  b_1\\
	  	  a_{21} & \dots & a_{2n}  & b_2 \\
	  	  \vdots  & \ddots & \vdots & \vdots \\
	  	  a_{m1} & \dots & a_{mn} & b_m
	  \end{pmatrix}
\end{align*}

Эта матрица содержит всю информацию о системе уравнений, за исключением обозначений неизвестных.

\emph{Теорема Кронекера-Капелли}. Система $ A x = b $ \emph{совместна} (т.е. имеет хотя бы одно решение) тогда и только тогда, когда ранг матрицы системы равен рангу расширенной матрицы $ \rg A = \rg (A \ | \ b) $.

Если $ \rg A \neq \rg (A \ | \ b) $, то система несовместна -- не имеет решений.

Если система имеет решение, то столбец свободных членов есть линейная комбинация столбцов матрицы системы. Поэтому при вычеркивании столбца $ b $ из расширенной матрицы $ (A \ | \ b) $ ее ранг не изменяется. Следовательно, $ \rg (A \ | \ b) = \rg A $.

ЗАМЕЧАНИЕ: теорема Кронекера-Капелли дает лишь критерий существования решения системы, но не указывает способа отыскать этого решения.


\section{Общее решение системы линейных алгебраических уравнений}

Неизвестные, которым соответствуют столбцы, входящие в базисный минор, называются \emph{базисными переменными}, остальные неизвестные -- \emph{свободными переменными}.

\emph{Общее решение} системы, выржающее базисные переменные через свободные, имеет вид \cite[\strbook{192}]{bortakovskiy:2005}
\begin{align*}
	\begin{cases}
		x_1 = b_1^{'} - a_{1, r + 1}^{'} x_{r + 1} - \ldots - a_{1, n}^{'} x_n, \\
		\ldots \\
		x_r = b_r^{'} - a_{r, r + 1}^{'} x_{r + 1} - \ldots - a_{r, n}^{'} x_n,
	\end{cases}
\end{align*}
где $ x_1, x_2, \ldots, x_r $ -- базисные переменные; $ x_{r + 1}, x_{r + 2}, \ldots, x_{n} $ -- свободные переменные.

\emph{Частное решение} системы -- решение системы, получающееся из общего решения, заданием конкретных значений свободными переменным.

Пусть $ x^{н} $ -- решение неоднородной системы. Тогда любое решение $ x $ неоднородной системы можно представить в виде $ x = x^\text{н} + x^\text{о} $, где $ x^\text{о} $ -- решение однородной системы.

Говорят, что \emph{общее решение} неоднородной системы есть сумма \emph{частного решения} \underline{неоднородной} системы и \emph{общего решения} соответствующей \underline{однородной} системы \cite[\strbook{200}]{bortakovskiy:2005}
\begin{align*}
	x = x^\text{н} + C_1 \varphi_1 + C_2 \varphi_2 + \ldots + C_{n - r} \varphi_{n - r}.
\end{align*}


\section{Решение систем уравнений с помощью полуобратных матриц}

Требуется решить систему линейных уравнений
\begin{align*}
	A x = b,
\end{align*}
где $ A $ -- \underline{произвольная} матрица размера $ m \times n $.

Если матрица системы нулевая $ A = O $, то система либо несовместна (при $ b = o $), либо имеет бесконечное множество решений (при $ b = o $ любой подходящий по размерам столбец $ x $ является решением). Далее рассматривается случай ненулевой матрицы $ A $.

Пусть $ A^{\neg 1} $ -- матрица, полуобратная к матрице системы $ A $. Используя определение полуобратной матрицы, неоднородную систему $ Ax = b $ можно переписать так
\begin{align*}
	A A^{\neg 1} A x = b.
\end{align*}

Если $ x $ -- решение системы, то подставляя $ A x = b $ в левую часть последнего соотношения
\begin{align*}
	A A^{\neg 1} A x = b, \quad \rightarrow \quad A A^{\neg 1} b = b.
\end{align*}

Тогда
\begin{align*}
  (E_m - A A^{\neg 1}) \, b = o.
\end{align*}

Это необходимое и достаточное условие совместности системы.

Решением системы будет $ x = A^{\neg 1} b $. Но поскольку \emph{полуобратная матрица} определена \emph{неоднозначно}, то эта формула фактически задает множество решений системы. Преобразуем так, чтобы была видна структура этого множества, в частности, выявим количество независимых параметров
\begin{align*}
  A^{\neg 1}_0 = T \Lambda^T S = T \,
    \begin{pmatrix}
          \begin{array}{c | c}
          	E_r & O \\
          	\hline
          	O & O
          \end{array}
    \end{pmatrix}
    S,
\end{align*}
где $ S $ и $ T $ -- элементарные матрицы порядков $ n $ и $ m $ соответственно, $ \Lambda $ -- матрица простейшего вида, эквивалентная матрице $ A $ ($ \Lambda \sim A $), $ \rg A $.

\emph{Теорема о совместности неоднородной системы и о структуре ее общего решения}. Неоднородная система $ A x = b $ \emph{совместна} тогда и только тогда, когда столбец свободных членов является решением однородной системы $ \Psi b = o $. Если система $ A x = b $ совместна, то ее общее решение имеет вид \cite[\strbook{205}]{bortakovskiy:2005}
\begin{align*}
  x = x^\text{н} + x^\text{о} = A_0^{\neg 1} \, b + \Psi \, c = T
  \begin{pmatrix}
  \begin{array}{c | c}
    E_r & O \\
    \hline 
    O & O
  \end{array}
  \end{pmatrix}
  S \, b + T
  \begin{pmatrix}
    \begin{array}{c}
    	O \\
    	\hline
    	E_{n - r}
    \end{array}
  \end{pmatrix}
  c, \quad
  \Psi =
  \begin{pmatrix}
  	\begin{array}{c | c}
  		O & E_{m - r} S
  	\end{array}
  \end{pmatrix},
\end{align*}
где $ T, S $ -- элементарные преобразующие матрицы, $ c = (C_1 \ldots C_{n - r})^T $ -- столбец произвольных постоянных.

Алгоритм применения полуобратной матрицы:
\begin{enumerate}
	\item Привести матрицу $ A $ системы $ A x = b $ к простейшему виду: $ \Lambda = S A T $. При этом находятся элементраные преобразующие матрицы $ S $ и $ T $, а также ранг $ r = \rg A \geqslant 1 $.
	
	\item Проверить условие совместности системы $ \Psi b = o $. При $ r = m $ система совместна. Если $ r < m $, то составить матрицу $ \Psi = (O \, | \, E_{m - r}) \, S $ и проверить условие $ \Psi b = o $. Если условие выполняется, то система совместна. В противном случае система несовместна и процесс решения заканчивается.
	
	\item Найти частное решение неоднородной системы по формуле
	\begin{align*}
      x^\text{н} = A_o^{\neg 1} \, b = T
        \begin{pmatrix}
        	\begin{array}{c | c}
        		E_r & O \\
        		\hline
        		O & O
        	\end{array}
        \end{pmatrix}
        S \, b
	\end{align*}

    \item Составить фундаментальную матрицу
    \begin{align*}
    	\Phi = T 
    	\begin{pmatrix}
    		\begin{array}{c}
    			O \\
    			\hline
    			E_{n - r}.
    		\end{array}
    	\end{pmatrix}
    \end{align*}

    \item Записать общее решение системы в виде
    \begin{align*}
    	x = x^\text{н} + \Phi \, c,
    \end{align*}
    где $ c = (C_1 \ldots C_{n - r})^T $ -- столбец произвольных постоянных.
\end{enumerate}


\section{Псевдорешения системы линейных уравнений}

Система $ m $ линейных алгебраических уравнений с $ n $ неизвестными $ A x = b $ может иметь единственное решение, бесконечно много решений или вообще не иметь решений. Нужно изменить понятие решения так, чтобы любая система линейных уравнений имела бы единственное в некотором смысле <<решение>>.

Поставим каждому столбцу в соответсвие неотрицательное действительное число, а именно норму (модуль)
$$
| x | = \Big(\sum\limits_{i=1}^{n} x_i^2 \Big)^{1/2}.
$$

\emph{Псевдорешением} системы линейных уравнений называется наименьший по норме столбец $ \tilde{x} $ среди всех столбцов, минимизирующих величину $ |A x - b | $.

ЗАМЕЧАНИЕ: \emph{любая} система имеет единственное псевдорешение \cite[\strbook{209}]{bortakovskiy:2005}
\begin{align*}
	\tilde{x} = A^{\sim 1} b,
\end{align*}
где $ A^{\sim 1} $ -- псевдообратная матрица для матрицы системы.

Понятие псевдорешения позволяет обойти не только факт неединственности, но и факт несуществования решений.

Если система несовместна, то псевдорешение $ \tilde{x} $ обеспечивает наименьшую величину погрешности $ \varepsilon(x) = | A x - b | $.

Если система совместна, то псевдорешение $ \tilde{x} $ является ее решением, т.е. $ \varepsilon(\tilde{x}) = 0 $, причем наименьшим по норме.

Алгоритм нахождения псевдорешения неоднородной системы:
\begin{enumerate}
	\item Найти псевдообратную матрицу $ A^{\sim 1} $.
	
	\item Найти псевдорешение $ \tilde{x} = A^{\sim 1} b $.
\end{enumerate}

ЗАМЕЧАНИЕ: \emph{полуобратная} матрица определена \underline{неоднозначно} и потому задает не конкретное решение, а \emph{множество решений} системы. \emph{Псевдорешение}, полученное с помощью псевдообратной матрицы, всегда вычисляется в \emph{конкретное решение}.


\section{Свойства решений однородной системы}

Общее решение однородной системы $ Ax = o $ имеет вид \cite[\strbook{194}]{bortakovskiy:2005}
\begin{align*}
	\begin{cases}
		x_1 = -a_{1, r + 1}^{'} x_{r + 1} - \ldots - a_{1,n}^{'}x_n,\\
		{\centering \ldots} \\
		x_r = -a_{r, r + 1}^{'} x_{r + 1} - \ldots - a_{r,n}^{'} x_n.
	\end{cases}
\end{align*}

Некоторые свойства:
\begin{itemize}
	\item Если столбцы $ \varphi_1, \varphi_2, \ldots, \varphi_k $ -- решения однородной системы уравнений, то любая их линейная комбинация $ \alpha_1 \, \varphi_1 + \alpha_2 \, \varphi_2 + \ldots + \alpha_k \, \varphi_k $ также является решением однородной системы,
	
	\item Если ранг матрицы однородной системы равен $ r $, то система имеет $ (n - r) $ \emph{линейно независимых решений}.
\end{itemize}

Любая совокупность $ (n - r) $ линейно независимых решений $ \varphi_1, \varphi_2, \ldots, \varphi_{n - r} $ однородной системы называется \emph{фундаментальной системой решений}.

\emph{Теорема об общем решении однородной системы}. Если $ \varphi_1, \varphi_2, \ldots, \varphi_{n - r} $ -- фундаментальная система решений однородной системы уравнений, то столбец
\begin{align}\label{eq:ordinsys}
	x = C_1 \varphi_1 + C_2 \varphi_2 + \ldots + C_{n - r} \varphi_{n - r}
\end{align}
при любых значениях произвольных постоянных $ C_1, C_2, \ldots, C_{n - r} $ также является решением системы $ A x = o $, и, наоборот, для каждого решения $ x $ этой системы найдутся такие значения произвольных постоянных $ C_1, C_2, \ldots, C_{n - r} $, при которых это решение $ x $ удовлетворяет равенству \eqref{eq:ordinsys}.

\section{Функциональные матрицы скалярного аргумента}

\emph{Функциональной матрицей скалярного аргумента} $ t $ называется матрица, элементы которой являются функциями независимой переменной $ t $
\begin{align*}
	\underset{m \times n}{A(t)} = [\, a_{ij}(t) \,]_{i,j=1}^{m,n}
\end{align*}

Производная функциональной матрицы
\begin{align*}
	\underset{m \times n}{\dfrac{d A(t)}{dt}} = \Big[ \dfrac{d a_{ij}(t) }{dt} \Big]_{i,j=1}^{m, n}.
\end{align*}

Производная обратной матрицы (если она существует)
\begin{align*}
	\underset{m \times n}{ \dfrac{d A^{-1}(t)}{dt} } = - A^{-1}(t) \,\dfrac{d A(t)}{dt} \, A^{-1}(t).
\end{align*}

Производная определителя квадратной матрицы $ A(t) $ $ n $-ого порядка
\begin{align*}
	\dfrac{d}{dt} \det A(t) = \sum_{i=1}^{n} \sum_{j=1}^{n} A_{ij}(t) \dfrac{d a_{ij}(t)}{dt} = \tr \Bigg[ A^{+}(t) \, \dfrac{d A(t)}{dt} \Bigg],
\end{align*}
где $ A_{ij}(t) $ -- алгебраическое дополнение элемента $ a_{ij}(t) $ матрицы $ A(t) $; $ A^{+}(t) $ -- присоединенная матрица.


% Источники в "Газовой промышленности" нумеруются по мере упоминания 
\begin{thebibliography}{99}\addcontentsline{toc}{section}{Список литературы}
	\bibitem{bortakovskiy:2005}{\emph{Бортаковский А.С.} Линейная алгебра в примерах и задачах. -- М.: Высш. шк., 2005. -- 591~с.}
	
	\bibitem{gmurman:1972}{\emph{Гмурман В.Е.} Теория вероятностей и математическая статистика. -- М.: Высшая школа, 1972.~-- 368~с. }
	
	\bibitem{lagutin:2009}{\emph{Лагутин М.Б.} Наглядная математическая статистика. -- М.: БИНОМ, 2009.~-- 472~с. }
	
	\bibitem{kobzar:2012}{\emph{Кобзарь А.И.} Прикладная математическая статистика. Для инженеров и научных работников. -- М.: ФИЗМАТЛИТ, 2012.~-- 816~с. }
\end{thebibliography}

\end{document}
