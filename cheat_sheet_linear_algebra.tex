\documentclass[%
	11pt,
	a4paper,
	utf8,
	%twocolumn
		]{article}	

\usepackage{style_packages/podvoyskiy_article_extended}


\begin{document}
\title{Сборник заметок по линейной алгебре и сопряженным вопросам}

\author{\itshape Подвойский А.О.}

\date{}
\maketitle

\thispagestyle{fancy}


%\shorttableofcontents{Краткое содержание}{1}


\tableofcontents

\section{Система $ m $ линейных алгебраических уравнений с $ n $ неизвестными}

Матричная запись неоднородной системы уравнений имеет вид
\vspace*{-3mm}
\begin{align*}
	A x = b,
\end{align*}

\vspace*{-3mm}
а однородной
\vspace*{-3mm}
\begin{align*}
	A x = o,
\end{align*}

\vspace*{-3mm}
где $ o $ в правой части обозначает нулевой столбец размеров $ m \times 1 $.

Эту матричную запись неоднородной системы уравнений можно представить в эквивалентной форме
\begin{align*}
	\begin{pmatrix}
		a_{11} \\
		\vdots \\
		a_{m1}
	\end{pmatrix} x_1 + 
    \begin{pmatrix}
    	a_{12} \\
    	\vdots \\
    	a_{m2}
    \end{pmatrix} x_2 + \ldots +
    \begin{pmatrix}
    	a_{1n} \\
    	\vdots \\
    	a_{mn}
    \end{pmatrix} x_n = 
    \begin{pmatrix}
    	b_1 \\
    	\vdots \\
    	b_m.
    \end{pmatrix}
\end{align*}

Тогда решение системы представляется столбцом
\begin{align*}
	x =
	\begin{pmatrix}
		\alpha_1 \\
		\vdots \\
		\alpha_n
	\end{pmatrix}
\end{align*}
и удовлетворяте равенству
\begin{align*}
	\begin{pmatrix}
		a_{11} \\
		\vdots \\
		a_{m1}
	\end{pmatrix} \alpha_1 +
    \begin{pmatrix}
    	a_{12} \\
    	\vdots \\
    	a_{m2}
    \end{pmatrix} \alpha_2 + \ldots + 
    \begin{pmatrix}
    	a_{1n} \\
    	\vdots \\
    	a_{mn}
    \end{pmatrix} \alpha_n =
    \begin{pmatrix}
    	b_1 \\
    	\vdots \\
    	b_m,
    \end{pmatrix}
\end{align*}
т.е. столбец свободных членов $ b $ является линейной комбинацией столбцов матрицы системы.




\section{Теорема (правило) Крамера}

Система называется \textbf{совместной}, если она имеет \emph{хотя бы одно решение}. Система называется \textbf{несовместной}, если она \emph{не имеет ни одного решения}.

Если определитель $ \Delta = \det A $ матрицы системы $ n $ линейный независимых уравнений с $ n $ неизвестными отличен от нуля ($ \det A \neq 0 $), то система имеет \emph{единственное} решение, которое находится по формулам
\begin{align*}
	x_i = \dfrac{ \Delta_i }{ \Delta }, \ i = 1,\ldots, n, \quad (\Delta = \det A \neq 0),
\end{align*}
где $ \Delta_i $ -- определитель матрицы , полученной из матрицы системы $ A = [a_{ij}]_{i,j=1}^n $ заменой $ i $-ого столбца столбцом свободных членов.

ЗАМЕЧАНИЕ: на практике при больших $ n $ правило Крамера не применяется!

Если $ \Delta = 0 $ (матрица коэффициентов системы вырождена) и хотя бы один определитель $ \Delta_i \neq 0 $, то система \emph{несовместна}, т.е. не имеет ни одного решения. Если же $ \Delta = \Delta_1 = \Delta_2 = \ldots, \Delta_n = 0 $, то возможны два случая: либо система несовместна (не имеет ни одного решения), либо система имеет бесконечно много решений \cite[\strbook{188}]{bortakovskiy:2005}.

\section{Условие совместности системы линейных уравнений. Теорема Кронекера-Капелли}

Рассмотрим систему $ m $ линейных уравнений с $ n $ неизвестными. Составим блочную матрицу, приписав к матрице $ A $ справа столбец свободных членов $ b $. Получим \emph{расширенную матрицу системы}
\begin{align*}
	\underset{m \times (n + 1)}{(  A \ | \ b )} =
	  \begin{pmatrix}
	  	  a_{11} & \dots & a_{1n} &  b_1\\
	  	  a_{21} & \dots & a_{2n}  & b_2 \\
	  	  \vdots  & \ddots & \vdots & \vdots \\
	  	  a_{m1} & \dots & a_{mn} & b_m
	  \end{pmatrix}
\end{align*}

Эта матрица содержит всю информацию о системе уравнений, за исключением обозначений неизвестных.

\emph{Теорема Кронекера-Капелли}. Система $ A x = b $ \emph{совместна} (т.е. имеет хотя бы одно решение) тогда и только тогда, когда ранг матрицы системы равен рангу расширенной матрицы $ \rg A = \rg (A \ | \ b) $.

Если система имеет решение, то столбец свободных членов есть линейная комбинация столбцов матрицы системы. Поэтому при вычеркивании столбца $ b $ из расширенной матрицы $ (A \ | \ b) $ ее ранг не изменяется. Следовательно, $ \rg (A \ | \ b) = \rg A $.




% Источники в "Газовой промышленности" нумеруются по мере упоминания 
\begin{thebibliography}{99}\addcontentsline{toc}{section}{Список литературы}
	\bibitem{bortakovskiy:2005}{\emph{Бортаковский А.С.} Линейная алгебра в примерах и задачах. -- М.: Высш. шк., 2005. -- 591~с.}
	
	\bibitem{gmurman:1972}{\emph{Гмурман В.Е.} Теория вероятностей и математическая статистика. -- М.: Высшая школа, 1972.~-- 368~с. }
	
	\bibitem{lagutin:2009}{\emph{Лагутин М.Б.} Наглядная математическая статистика. -- М.: БИНОМ, 2009.~-- 472~с. }
	
	\bibitem{kobzar:2012}{\emph{Кобзарь А.И.} Прикладная математическая статистика. Для инженеров и научных работников. -- М.: ФИЗМАТЛИТ, 2012.~-- 816~с. }
\end{thebibliography}

\end{document}
